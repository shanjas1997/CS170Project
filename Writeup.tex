\documentclass[12pt]{article}
\usepackage[a4paper, total={5.5in, 9in}]{geometry}
\usepackage{amsmath}
\DeclareMathOperator*{\argmax}{arg\,max}
\title{%\vspace{-1.5cm}            % Another way to do
CS170 GargSack WriteUp}
\author{Jason Shan, Dhruv Malik}
\begin{document}
\maketitle
\indent The objective of this algorithm is to give an approximation to a set of problems that may or may not have optimal solutions. Due to the size of these files, we want to run our solution in such a way so that we do not break the constraints of memory or risk timing out. Thus, our proposal will to run a few different greedy strategies sorted by different heuristics, and see what our maximum value will be. \\
\indent To make things non-deterministic, we will introduce randomness to a few of our cases. Our cases are relatively simple. They are:
\begin{itemize}
   \item Randomize classes, and then pick our classes in such a way so that we do not break runtime constraints. We will run this many times to try to, by chance, get the most-optimal set we can. Once we get this set, we know that nothing in this set will conflict in terms of classes, so we can run a simple greedy on this.
   \item Run greedy algorithm based on our multiple sorting heuristics, that involve some subset of weight, cost, resell, class, etc.
\end{itemize}
\end{document}